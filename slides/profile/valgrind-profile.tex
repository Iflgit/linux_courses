\begin{frame}[fragile]
 \frametitle{Valgrind callgrind}
 \begin{itemize}
  \item Инструмент для профилирования \verb+ --tool=callgrind+
  \item Просмотр результатов
    \begin{itemize}
      \item \texttt{kcachegrind}
      \item \texttt{callgrind\_annotate}
    \end{itemize}
 \end{itemize}
% \begin{center}
%  Упражнение
% \end{center}
% \begin{enumerate}
%  \item Скомпилировать ту же самую программу с ключами \verb+-g+
%  \item Запустить под valgrind
%  \item Изучить callgraph и относительное время в функциях
% \end{enumerate}
\end{frame}


\begin{frame}[fragile]
    \frametitle{Упражнение}
    \begin{enumerate}
        \item Создаем рандомный файл с необходимым размером: {\tt dd if=/dev/urandom of=input bs=5M count=1}
        \item {\tt time gzip input -c >/dev/null}
        \item Запускаем с помощью valgrind: \\
	    {\tt time valgrind --tool=callgrind gzip input} \\
	    Ждем.
        \item Сравниваем результаты по времени.
        \item Смотрим появившийся файл callgrind.out.\$PID
        \item Запускаем: kcachegrind callgrind.out.\$PID
	\item Смотрим получившийся callgraph и относительное время в функциях.
        \item Запускаем: callgrind\_annotate callgrind.out.\$PID
        \item Добавить опции {\tt -{}-collect-jumps=yes -{}-dump-line=yes -{}-dump-instr=yes} для valgrind и запустить kcachegrind
    \end{enumerate}
\end{frame}
