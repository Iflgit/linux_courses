\documentclass[12pt,a4paper,oneside]{article}
\usepackage[left=0.5cm,right=0.5cm,top=2cm,bottom=2cm,bindingoffset=0cm]{geometry}

\usepackage[russian]{babel}
\usepackage[utf8]{inputenc}
\usepackage[T1]{fontenc}

\begin{document}

% Формируем титульную страницу
\title{Программа курсов по ОС Linux}
\author{Minsk Linux Users Group}

\maketitle

\renewcommand{\contentsname}{Оглавление}

\section{Программа}

\subsection{Введение в GNU/Linux}

\begin{itemize}
    \item Введение \\
        Обзор свободных лицензий. Принципы проектирования переносимых программ.
    \item ОС Linux \\
        Дистрибутивы. Процесс загрузки ОС Linux. Управление загрузкой.
    \item Командная строка \\
        Понятие shell, console. Окружение. Потоки ввода/вывода. Базовые утилиты. Vim. Sed.
    \item Управление ПО \\
        ``DLL Hell''. Пакетные менеджеры. Установка, обновление и удаление пакетов.
    \item Пользовательская модель \\
        Управление пользователями, группами.
    \item Файловая структура \\
        FHS. Файловые объекты. Права доступа. Утилиты.
    \item Дисковая подсистема \\
        Блочные устройства. Файловые системы. RAID. LVM.
    \item Сетевое администрирование \\
        Управление интерфейсами. Iptables. Удаленный доступ.
\end{itemize}

\subsection{Командная оболочка и язык сценариев Bash}

\begin{itemize}
    \item Интерактивный режим \\
        Коды возврата. Перенаправление ввода/вывода.  Экранирование.
    \item Синтаксис \\
        Переменные. Параметры. Тесты и сравнения. Ветвления. Циклы. Арифметические операции. Массивы. Манипуляции со строками.
    \item Сценарии \\
        Функции. Встроенные и внешние команды. Подстановка параметров. exec. Атрибуты переменных. Обработка сигналов. Отладка.
\end{itemize}

\subsection{Инструменты разработчика}
\begin{itemize}
    \item GNU Toolchain \\
        Компилятор. Линкер. Статические и динамические приложения и библиотеки.
    \item Управление сборкой проекта \\
        Make. Введение в синтаксис makefile. Autotools.
    \item Базовый инструментарий \\
        Работа с исходным кодом. Анализ исполняемого файла. Утилиты для анализа. Структура elf. Binutils.
    \item Отладка \\
        Трассировка. GNU отладчик gdb. Дампы.
    \item Профайлинг \\
        Определение ``узких мест''. Valgrind. Perf.
    \item Покрытие исходного кода \\
        gcov.
    \item Системы документирования исходного кода \\
        Doxygen.
    \item Совместная разработка \\
        Централизованные и распределенные системы контроля версий. SVN. GIT. История изменений. Бранчи, слияния, тэги.
    \item Установка ПО \\
        Установка из исходников. Введение в пакетирование RPM. Run-time и build-time окружения.
\end{itemize}

\end{document}
